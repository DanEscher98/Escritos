\documentclass{article}
\usepackage[hybrid]{markdown}   % Necesaria para incluir archivos md
\usepackage[utf8]{inputenc}	    % Importante para escribir acentos
\usepackage{indentfirst}        % Para indentar el primer parrafo
\usepackage{blindtext}          % Permite incorporar archivos externos
\usepackage{xifthen}            % Aqui se encuentra la macro \IfFileExists
\usepackage{tikz}               % Aqui se encuentra la macro \foreach
\usepackage{hyperref}
\usepackage{bookmark}
\usepackage[spanish]{babel}
\renewcommand{\markdownRendererHeadingTwo}[1]{\section*{#1}}
\renewcommand{\markdownRendererHeadingThree}[1]{\section*{#1}}
%Para compilar el documento
%pdflatex --shell-escape file.tex


\begin{document}
\begin{titlepage}
    \begin{center}
        \vspace*{1cm}
        \Huge\textbf{Markdown Example}\\
        \vspace{1.5cm}
        \textbf{Daniel Colin}
        \vfill   
        A simple template    
        \vspace{0.8cm}         
    \end{center}
\end{titlepage}
\tableofcontents

\chapter*{Introducción}
\foreach \Month in {Ene,Feb,Mar,Abr,May,Jun,Jul,Ago,Sep,Oct,Nov,Dic}{
    \chapter{\Month}\newpage
    \foreach \Day in{1,...,31}{
        \IfFileExists{2021/\Month/day-\Day.md}{
            \markdownInput{2021/\Month/day-\Day.md}
        }
        {}
    }
}

\end{document}
