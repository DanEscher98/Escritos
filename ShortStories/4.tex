\mytitle{\Large Por entre las corrientes doradas}\\[0.1cm]
\marginnote{\it $\sim$ A Pechi}

Dos siluetas son las que caminan. Una lleva sombrero, la otra un bolso de algas tejidas. Sus pies descalzos dejan tenues huellas en el manto de hojas.

- "Quizás debí hacerte caso y traer el quinqué en lugar de esto. Había entonces un brillante sol. ¿Cómo lo iba a saber yo? En un momento se tornó de tinta. Es el cielo un calamar tan miedoso."

- "Tal vez, pero entonces no podrías haberme prestado tu sombrero. Para cubrirme el rostro, para dormir apacible en aquel prado. Toma, llevo guardadas varias hogazas todavía. Siguen crujientes y aún les giran sus manecillas."

- "¿Fue feliz tu sueño? Se me hace tan largo, es ya indistinto dormir de día o andar de noche. ¿Pudiste ver el fin de nuestro andar? Al menos las luciérnagas nadan fielmente a nuestro lado. (...) Gracias por el tiempo."

Imponentes torres los rodean, fumarolas marinas de cuyas humeantes volutas caen apacibles hojas. Hambre no es lo que tenía, ni siquiera real cansancio. Tan sólo la incertidumbre de si algún día llegarían. Saboreó el momento. Era su preferido, tenía pasas. Estaba endulzado con miel y había sido horneado a fuego lento. Alzó la mirada, miró a su costado. Tan oceánico abismo podía ser la noche.

- "Creo que olvidé el sueño. Pero era feliz pues mirábamos al unísono. Oh, era a través de la claraboya de un navío naufragado, que hacia el horizonte dirigíamos la mirada. A lo lejos un brillante recuerdo nos llamaba. Pero, no recuerdo más. ¿Acaso guardabas el vaivén de un péndulo en mi bolso?"

- "Si tan sólo pudiesemos regresar a la cabaña. Hornearíamos más de éstos. Y habría traido con qué iluminar nuestro paso. Tendríamos con qué espantar este cobijo tan oscuro. Seríamos una relumbrante estrella, ignorante de las olas sobre nuestras cabezas. Quisiera ya llegar."

- "¿Pero no ves que la vigilia no sería más que este dormir? Si hemos de cruzar que sea a paso quedo. Hollando la memoria con nuestro andar. Alimentando la esperanza con nuestras huellas. ¿No ves que tu dorado sombrero sostiene el rocío?"

Tanto han caminado que la luna está ahora a sus espaldas y una cornalina tienen entre sus manos.\\[0.5cm]