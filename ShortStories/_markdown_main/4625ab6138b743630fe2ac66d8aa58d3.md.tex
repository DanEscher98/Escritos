\markdownRendererHeadingOne{Mosaico en cuatro estampas}\markdownRendererInterblockSeparator
{}"En un olvidado conflicto bélico, el amado de una maestra se alista en el ejército y muere. Al final se da un vistazo a lo que hubiese pasado de no ir a la guerra, teniendo su familia."\markdownRendererInterblockSeparator
{}\markdownRendererHeadingTwo{Preludio Andante}\markdownRendererInterblockSeparator
{}Si en la costa de un país, en lugar de rocas y botes encallados hubiese arena, es seguro que sin dudarlo la ocuparían los oficiales con sus guirnaldas doradas. Porque cuando el cielo se encubre de sueño, ellos avanzan sigilosos dirigidos por un coronel francés. Buscan con avidez abrigo para sus pies gélidos de espera. Pero hay agua y la playa se baña en sus olas. El desierto se extiende y no tienen que ser dunas las que lo cubran. Los confundidos niños esperan a que acabe su clase de álgebra, mientras miran el mapa de Chile y su gran abismo llamado Atacama. El fruto es rojo, la nieve blanca y el mar azul. No hay razón para tener frío. \markdownRendererInterblockSeparator
{}\markdownRendererHeadingTwo{Minueto Adagio}\markdownRendererInterblockSeparator
{}La maestra toma con cuidado la carta que horas atrás, guardó con recelo en su cajón. Los niños se han ido y tiene ya ella un respiro. ¿Será posible que aquella paloma impertinente tocara a su ventana, justo cuando estaba por explicar el trinomio cuadrado? Caminando hacia casa, sus brazos bailan junto al péndulo del reloj en salón de clases. Sus pensamientos están corriendo, pero ella ha olvidado caminar. Incluso de mirar a la rosa entre las orquídeas se ha olvidado su andar. La paloma no debería volar en círculos en torno a ella, abatiendo sus sentimientos. Abriendo las alas y proyectando una sombra parda junto al jardín... diríase que es un buitre. \markdownRendererInterblockSeparator
{}\markdownRendererHeadingTwo{Antífona A cappella}\markdownRendererInterblockSeparator
{}-- "González, ¿se encuentra mejor?" -- "Sí señor, dentro de poco me habré repuesto y podré volver al frente. Como si aquellos desgraciados supieran apuntar." ¿Es tan banal el conflicto? Cuando la sangre es roja, los huesos siempre ocultos en la carne mantienen su blanca pureza, y la pluma con su danza azul decide nombrar a la amada. Nada vale más que el sencillo trago de agua de manantial, cuando la gangrena interna obliga a sentir el flujo de la vida, bramante como río en su caudal. El coronel salió de la carpa, pero no sin antes haber cerrado los ojos de su oficial. \markdownRendererInterblockSeparator
{}\markdownRendererHeadingTwo{Lied Ad libitum}\markdownRendererInterblockSeparator
{}-"El caballo quiere trotar papá." -"No, debes dejarlo descansar. Que tú quieras volver a montarlo, no significa que él también quiera. Ven, quiero mostrarte algo." Y lo guía hacia un arcón en el establo que el pequeño nunca había visto. Lo abre y sus chirriantes goznes entonan el coro que precede al relato. ¿Qué sentido tiene indagar, explorar por la selva de lo que pudo haber sido? Mientras los remordimientos se retuercen tratando de romper sus cadenas, queriendo cambiar lo que fue. Se quiere manejar el destino de la vida misma, pero el desierto se interpone en tu travesía, extendiendo dunas heladas frente a ti. ¿De qué sirve ver al tintero junto a una medalla? Si ambos empolvados y fríos son incapaces de consolar al corazón doliente. - "¡Vengan, es hora de comer!" ¿De qué sirve? Si terminado el relato vuelven a su tumba, la tinta y su compañía.\markdownRendererInterblockSeparator
{}Un buitre custodiaba la comarca.\relax