\markdownRendererHeadingOne{El cántaro calcinado}\markdownRendererInterblockSeparator
{}-- "Son como rasguños sobre una pizarra, pero con timbre de gotas de lluvia."\markdownRendererInterblockSeparator
{}Se recargó un poco más en la silla y siguió viendo tras la ventana. Afuera estaba aún húmeda la calle, pero se combinaba con un hermoso tinte de alba. Convergía en la lejanía con la guarida de las escamas aladas. Tras él, se iba iluminado su repisa. La pupila celeste observaba su colección de porcelana china que, a pesar de estar quebrada y llena de tizne, conservaba sus hermosos colores.\markdownRendererInterblockSeparator
{}-- "Ni siquiera pude encender el agua para cocer el arroz. Me pregunto cómo llegó hasta aquí."\markdownRendererInterblockSeparator
{}Su teclado estaba demasiado sucio como para mecanografiar siquiera un sencillo recuerdo. Había olvidado su antiguo oficio para ser ahora huesa, de la que sin embargo, se erguían tímidos retoños. Estos nunca habían oído el volcánico bramido. Le venía a su mente, ahora desnuda de cráneo, la imagen de hombres encorvados sobre sembradíos encharcados. Y decidió no quitar el renuevo. Decidió dejar que el verde se nutriera de tierra entre las teclas negras.\markdownRendererInterblockSeparator
{}-- "Al menos puedo sentir como un calor, una cálida caricia frente al amanecer."\markdownRendererInterblockSeparator
{}En el noveno piso, el de las ventanas quebradas y pasillos vacíos, se pasea una tranquilidad descarnada. Intenta removerse las sábanas de encima, rasgar una uterina bolsa para nacer de nuevo a lo tangible. Le hubiera gustado ver sus propios restos, como para convencerse de que estaba muerto. Pero ya no había mas que polvo. Cenizas de crematorio. Ahora sólo le queda la esperanza de hallar el cómo grabar memorias con sus espectrales dedos de gasa y retoñar él también.\markdownRendererInterblockSeparator
{}-- "\markdownRendererEmphasis{¿Habrá tenido nombre el dragón?}"\relax