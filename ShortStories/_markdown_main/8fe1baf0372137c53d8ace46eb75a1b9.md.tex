\markdownRendererHeadingOne{Composición a cuatro manos}\markdownRendererInterblockSeparator
{}Subía dando tumbos entre tejados y azoteas, en virtud del inoportuno viento, una hoja. Más suave que el lirio, más triste que una nomeolvides. Volando entre espasmos rosados, resultaba cual mancha en el azulejo del cielo. Y abajo los hombres que pasaban por buenos. Una gaviota aleteó a su lado y cayó con su piel rosada en el alféizar de un segundo piso. Al lado de la ventana, un señor canoso nota su llegada esbozando una sonrisa. Toma otra hoja, parienta suya pero color crema, y escribe. Oprimiéndola entre la pluma y el barniz ajado de la mesa.\markdownRendererInterblockSeparator
{}-- "¡Baja por favor!" le dice su esposa. Tan rápido se le había ido el tiempo que ya era la cena. Baja los veintiún escalones de madera, pero está distraído.\markdownRendererInterblockSeparator
{}A la mesa llega una florecita, su madre la regaña y pretende sonreir, pero no puede. Se disculpa, pues otra vez ha llegado tarde. Acomoda su talle de olanes verdes en la silla de al lado. Él se sienta también a la mesa. Tres son las sillas ahora ocupadas.\markdownRendererInterblockSeparator
{}-- "Querido, se te enfriará el café. Ya tiene miel, como te gusta."\markdownRendererInterblockSeparator
{}-- "Sí, sí, perdona. Era sólo ..." dijo antes de soplar y dar un sorbo a su taza; "... bueno ... nada. Creo que mañana iré al centro por más tinta."\markdownRendererInterblockSeparator
{}Su padre sigue escuchándola. Hace ya cuatro semanas, y no puede quitarse esa extraña sensación. Estaban, cuando los vio, no tan juntos, sentados en una banca bajo el árbol. No los oyó, porque al ir con ellos, pararon, turbándose. Y al besar a su hija, los miro y noto en sus ojos una quietud esperanzada, que no pudo más que alterarlo. Si tan solo hablase más y a fondo con su madre. Si es que fuese que él, como padre, le incomodaba. O si fuesen ya novios y pudiese dar francamente su opinión. Pero nada, ni lo menciona por nombre, como si no fuese nada. Tiene veinte años y es como una cereza en almíbar. Demasiado dulce y metida siempre en su jarrón. Ahora se arrepiente de no haberla enviado a las misiones, o a servir en los orfanatos.\markdownRendererInterblockSeparator
{}Su hija se apresura a comentar sobre la salud del gorrioncito de la vecina, que es muy amiga suya. Él hace lo mejor que puede como padre y la escucha mientras comen. Hace ya cuatro semanas y no puede quitarse esa extraña sensación. Cuando las vio, no estaban tan cerca. Estaban sentadas en una banca cerca del árbol. No alcanzó a escuchar lo que decían, pues al ir hacia donde estaban se pararon con turbación. Las saludó y se acercó para besar como siempre en la frente a su hija. Al ver sus ojos, notó en ellos una quietud esperanzada, que no pudo más que alterarlo. Si tan sólo ya hablase más a fondo con su madre. Si es que él, como padre, le incomodaba. O si hubiesen anunciado su noviazgo y pudiese dar francamente su opinión. Pero nada, ni lo mencionaba como si no fuese nada. Tiene veinte años y es como una cereza en almíbar. Demasiado dulce y metida siempre en su jarrón. Ahora se arrepentía de no haberla enviado a las misiones o a servir en los orfanatos. Acaso sería ahora fruto y se le habría tostado el corazón.\markdownRendererInterblockSeparator
{}Caía la noche. Habiendo colocado sus hojas de nuevo en la gaveta, se disponía a disfrutar los últimos rayos desde su mecedora. El pueblo que lo rodeaba solía llamar con su bullicio en la plaza a los jóvenes, en especial aquel día que era jueves. En otros tiempos, él también se habría dirigido hacia allá. Alistado su puesto de tiro al blanco, con peluches grandes para quien rompiera más botellas. Pero de eso ya hacía tiempo. Habiendo vendido casi todo ahora sólo se mecía. El acostumbrado chirrido con el vaivén de las pequeñas olas costeras. El suave balanceo del viento y una pequeña fuga en la proa que no importa pues ha encallado en la arena. Se levanta, coloca los remos a un lado y toma el rifle, el único que no vendió de los que tenía. Echó un leve vistazo a lo largo de la playa, enfrente se levantaban árboles diversos. No era muy claro, pero la luna sugería un camino. Esta era una de las islas más alejadas que formaban el archipiélago en el que se encontraba, hasta ahora la más lejana a la que se aventuraba. Con cuidado y atento a los ruidos, se adentró.\markdownRendererInterblockSeparator
{}A la mañana siguiente, su esposa seguía acostada a su lado, de nuevo él le había ganado al despertador. Decidió que regresaría con la tinta para antes del desayuno. Con su paso calmado, en media hora llegaba al centro. Se arregló en silencio y salió al pasillo que daba a las escaleras. Poniendo una mano en el barandal vio para atrás antes de bajar, al fondo se hallaba la puerta del dormitorio de su hija. Bajó.\markdownRendererInterblockSeparator
{}Discreta mancha rosa, que apresurada y nerviosa, llegaba a tapar el sol. Corría y caía, mas el aire oía su llanto llevándola lejos.\markdownRendererInterblockSeparator
{}Sonrosada como muchas de la plaza y tan apabullada en pétalos, como una carmelita.\relax